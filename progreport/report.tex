
\documentclass[12pt,a4paper,twoside]{article}
\usepackage[pdfborder={0 0 0}]{hyperref}
\usepackage[margin=17mm]{geometry}
\usepackage{graphicx}
\usepackage{amsfonts}
\newcommand{\M}{\mathcal{M}}
\begin{document}

\centerline{\Large Computer Science Tripos Project Progress Report}
\vspace{0.35in}
\centerline{\LARGE\bf Dynamic Partial-Order Reduction for Model Checking}
\vspace{0.3in}
\centerline{\large Isaac Dunn, Clare College}
\vspace{0.1in}
\centerline{\href{mailto:ird28@cam.ac.uk}{ird28@cam.ac.uk}}
\vspace{0.2in}
\centerline{\large 28 January 2015}

\vfil


\noindent
{\bf Project Supervisors:} Dr. Jonathan Hayman \& Prof. Glynn Winskel
\vspace{0.2in}

\noindent
{\bf Director of Studies:} Prof. Lawrence Paulson
\vspace{0.2in}
\noindent
 
\noindent
{\bf Project Overseers:} Dr. Stephen Clark  \& Prof. Alan Mycroft


% Main document

\section*{Work Completed}

A small ML-like language for the expression of
concurrent programs has been designed, implemented and tested.
In the language, multiple threads execute asynchronously,
communicating via shared objects, for which an atomic \emph{compare-and-swap} operation is provided.

A simple backtracking model checking algorithm has been implemented, which
explores each possible execution path in turn. This is used to verify that
the Dynamic Partial-Order Reduction (DPOR) algorithm gives the correct answers,
and it also gives a reference to which the performance of that algorithm can be
compared.

The DPOR algorithm \cite{popl05} has been successfully implemented, fulfilling
a significant part of the project's main success criterion, and putting the
project one or two weeks ahead of the original work schedule. Pleasing performance
improvements have been measured on a number of example programs that have
been tested so far, suggesting that the project's main success criterion will be
met once more meaningful performance analysis is done.

A program has been written which generates graphs showing the reduction in the
size of the explored state space achieved by the DPOR algorithm (see
the Appendix for an example).
This was not part of the original proposal, but serves well to illustrate
my work.


\section*{Unexpected Difficulties}

No serious unexpected difficulties have arisen. A few non-trivial bugs
have made an appearance so far, but this is of course to be expected,
and they have been resolved without issue.

\section*{Future Work}

To fully meet the main success criterion of the project, a quantitative
comparison will be made between the performances of the simple and the DPOR
model checking algorithms. Beyond this, extensions to the project can be
made, including implementing and evaluating ameliorations to the DPOR
algorithm, adding
mutual-exclusion locks to the language\footnote{Although it is true
	that \texttt{lock(L)} is equivalent to \texttt{while not cas(L, false, true)}, the latter introduces a cycle to the state space, which is fatal for model checking, whereas inbuilt support for the former
	does not.}, or implementing and comparing another model checking algorithm which attempts to address the state explosion problem.

\begin{thebibliography}{9}
	
	\bibitem{popl05}
	\href{https://users.soe.ucsc.edu/~cormac/papers/popl05.pdf}{
		C. Flanagan and P. Godefroid,
		\emph{Dynamic Partial Order Reduction
			for Model Checking Software},
		Proceedings of the 32nd Symposium on Principles of Programming Languages,
		2005.
	}

\end{thebibliography}

\appendix

\section*{Appendix -- Example Graph}

\begin{figure}[tbh]
	\centerline{\includegraphics[width=\textwidth,keepaspectratio]{egfig.png}}
	\caption{Graph showing which states (in grey) and transitions
		(dashed lines) are never explored by the DPOR algorithm
		on an example program}

\end{figure}

\end{document}
