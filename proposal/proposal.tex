
\documentclass[12pt,a4paper,twoside]{article}
\usepackage[pdfborder={0 0 0}]{hyperref}
\usepackage[margin=25mm]{geometry}
\usepackage{graphicx}
\usepackage{amsfonts}
\newcommand{\M}{\mathcal{M}}
\begin{document}

\vfil

\centerline{\Large Computer Science Tripos Project Proposal}
\vspace{0.4in}
\centerline{\LARGE\bf Symbolic Model Checking}
\vspace{0.4in}
\centerline{\large Isaac Dunn, Clare College}
\vspace{0.3in}
\centerline{\large 11 October 2015}

\vfil


\noindent
{\bf Project Supervisor:} To Be Confirmed
\vspace{0.2in}

\noindent
{\bf Director of Studies:} Prof. Lawrence Paulson
\vspace{0.2in}
\noindent
 
\noindent
{\bf Project Overseers:} Dr. Stephen Clark  \& Prof. Alan Mycroft


% Main document

\section*{Introduction}

When computer systems go wrong, the consequences can be disastrous.
As we rely on computers more and more, concern about such potential
failures will only increase, particularly in safety or mission critical applications.
This partially explains the increasing popularity of formal verification
methods, used alongside more traditional quality
assurance techniques, to show that (abstractions of)
hardware and software components meet their specifications. These methods
have proved capable of finding subtle errors that elude conventional
simulation and testing; concurrency race conditions, for example,
are notoriously difficult to detect.

One such method is known as model checking. The system in question
is described as a finite state machine, which is represented by a model $\M$ for
some appropriate temporal logic. Its specification is represented
by a formula $\phi$ in that logic. The verification
method consists of automatically computing whether the
model $\M$ satisfies $\phi$.

However, model checking typically suffers from the {\em state
explosion problem}: the number of states grows
exponentially with the number of components of the described system.
To mitigate this problem, McMillan \cite{mcmillan92}
proposes symbolic model checking: the use of Boolean functions
to represent sets and relations, thereby avoiding the explicit construction of
the state graph during the computation deciding $\M \models \phi$.
The aim of this project is to build a system which implements this technique.


\section*{Starting Point}

Although the project will be written from scratch, there will be nothing novel involved---published
algorithms and techniques will be used to implement the system.

\section*{Substance and Structure of the Project}

The project breaks down into the following three sub-tasks:

\begin{enumerate}
	\item Implement a module that uses Boolean functions (described
	by Ordered Binary Decision Diagrams) to represent sets and relations,
	including various necessary operations on them.
	
	\item Implement an algorithm which takes a model $\M$ for Computation
	Tree Logic (CTL) and a formula $\phi$
	and computes whether $\M \models \phi$ holds.
	
	\item Design a language for the description of systems and their
	specifications, and a compiler for
	it which translates system descriptions to an appropriate models $\M$, and
	translates specifications to appropriate formulae $\phi$.
\end{enumerate}

\section*{Main Success Criterion}

To demonstrate, through a few well-chosen simple examples, that the project
is capable of showing that systems do or do not meet their specifications.

\section*{Possible Extensions}

\begin{itemize}
	\item Support fairness constraints alongside CTL specifications.
	
	\item Compare performance with existing optimised symbolic model checkers
	such as NuSMV \cite{nusmv}.
	
	\item Ameliorate performance and quantitatively measure the improvement.
	
	\item If $\M \not\models \phi$ then provide a counterexample.
	
	\item Support specifications in Linear Time Logic (LTL) as well as CTL.
	
	\item Compare performance of the OBDD implementation with a na\"ive one.
\end{itemize}

\section*{Plan of Work}

\subsubsection*{Michaelmas Week 2}

Decide on an implementation language for the project. Study OBDDs and understand
how to effectively implement them.

\subsubsection*{Michaelmas Weeks 3--4}

Write, test and debug module that uses OBDDs to efficiently store and operate
upon sets (of states) and relations between them.

\subsubsection*{Michaelmas Weeks 5--6}

Study and implement the basic CTL model checking algorithm, making use of the
module written in the previous fortnight.

\subsubsection*{Michaelmas Weeks 7--8}

Use simple handwritten examples to test and debug the model checking algorithm.

\subsubsection*{Christmas Vacation}

Decide upon the syntax and semantics of the language that will be used to
facilitate the description and specification of systems. Implement a compiler
for that language which translates descriptions to CTL models and specifications
to CTL formulae.

\subsubsection*{Lent Weeks 1--2}

Use simple handwritten examples to test and debug the compiler. Write and submit progress report.

\subsubsection*{Lent Weeks 3--4}

Connect compiler to model checking algorithm, and test and debug whole system.
Main success criterion should be fulfilled by division of Lent term.

\subsubsection*{Lent Weeks 5--8}

Buffer time to catch up with timetable. If unneeded, attempt
extension tasks. Write preparation chapter of dissertation.

\subsubsection*{Easter Vacation}

Write implementation and evaluation chapters and submit dissertation.

\section*{Resources Required}

For this project, I shall mainly use my own computer,
an Acer laptop which runs Windows 10, has a 1.8 GHz dual core processor,
and 6 GB of memory. In the event of its failure, I will transition to using
the university's MCS machines. To mitigate the risk of loss of data,
I will make backups to my university filespace each day I work on the project,
and I will make weekly backups to both Dropbox and OneDrive.

\begin{thebibliography}{9}
	
	\bibitem{mcmillan92}
	\href{http://www.kenmcmil.com/pubs/thesis.pdf}{
		K. L. McMillan,
		\emph{Symbolic Model Checking} (PhD thesis),
		1992.
	}
	
	\bibitem{nusmv}
	\href{http://nusmv.fbk.eu/NuSMV/papers/cav99/html/index.html}{
		A. Cimatti, E. M. Clarke, F. Giunchiglia, M. Roveri,
		\emph{NuSMV: A New Symbolic Model Verifier},
		Proceedings of the 11th International Conference on Computer Aided Verification,
		1999.
	}
\end{thebibliography}

\end{document}
