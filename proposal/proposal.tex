
\documentclass[12pt,a4paper,twoside]{article}
\usepackage[pdfborder={0 0 0}]{hyperref}
\usepackage[margin=21mm]{geometry}
\usepackage{graphicx}
\usepackage{amsfonts}
\newcommand{\M}{\mathcal{M}}
\begin{document}

\centerline{\Large Computer Science Tripos Project Proposal}
\vspace{0.4in}
\centerline{\LARGE\bf Dynamic Partial-Order Reduction for Model Checking}
\vspace{0.4in}
\centerline{\large Isaac Dunn, Clare College}
\vspace{0.3in}
\centerline{\large 22 October 2015}

\vfil


\noindent
{\bf Project Supervisors:} Dr. Jonathan Hayman \& Prof. Glynn Winskel
\vspace{0.2in}

\noindent
{\bf Director of Studies:} Prof. Lawrence Paulson
\vspace{0.2in}
\noindent
 
\noindent
{\bf Project Overseers:} Dr. Stephen Clark  \& Prof. Alan Mycroft


% Main document

\section*{Introduction}

Model checking is category of formal methods used to show that systems
(such as hardware components and software modules) fit their specification.
Unlike standard testing or simulation, model checking ensures that the given
system will \emph{always} behave as expected. This is achieved by constructing
an abstract model of the system, and then automatically showing that this
model has properties which are equivalent to the system specification.

However, especially when verifying concurrent software, model checking suffers
from the \emph{state explosion problem}---the size of the state space of the model
grows exponentially with the number of components in the system. There have been
several attempts to address this problem, but this project will explore one in particular:
dynamic partial-order reduction.

Partial-order reduction attempts to reduce the size of the state space explored by
considering only a subset of the possible interleavings of events, such that any interleaving
not explored is, as far as the specification is concerned, equivalent to an interleaving
that is explored \cite{cgmp98}. However, it can be difficult to work out which sets of
interleavings are equivalent, so statically computed approximations of such sets are necessarily
conservative. In some cases, this can lead to only a modest reduction in the size of the
state space.

Dynamic partial-order reduction completes one full execution of the
program, picking an arbitrary interleaving of events, while dynamically making note
of backtracking points at which an alternative state transition leads to a path potentially not
``equivalent" to that already traversed. These paths are explored until all backtracking
points are exhausted; at this point, the program is shown to have met its specification
as if the entire state space had been explored \cite{popl05}. The extra information gained
from doing the computation dynamically means that often a significant reduction in the
size of the state space explored can be achieved. The aim of this project is to build a
system which carries out dynamic partial-order reduction model checking.

\section*{Starting Point}

The project will be written from scratch, although all the fundamental
theory and algorithms will be taken from published work.

\section*{Substance and Structure of the Project}

The project breaks down into the following three sub-tasks:

\begin{enumerate}
	\item Design and implement a simple language for the writing of
	concurrent software programs and the specification of simple
	safety constraints.
	
	\item Understand and implement the core algorithm which identifies
	the necessary backtracking points.
	
	\item Implement an algorithm which backtracks through the possible
	executions of the given program, using the above procedure to explore the
	interleavings of events which are necessary for soundness of verification.
\end{enumerate}

\section*{Main Success Criterion}

To demonstrate, through a few well-chosen simple examples, that the project successfully
performs dynamic partial-order reduction model checking. Quantitative comparison should
be made either with a model checking algorithm that does not use partial-order reduction
to show the improvement in performance, or ideally with an existing implementation of dynamic
partial-order reduction model checking to show that the performance is similar.

\section*{Possible Extensions}

\begin{itemize}
	\item Implement a static partial-order reduction model checker
	and compare.
	
	\item If the system does not meet its specification, provide a trace
	of a counterexample execution.
	
	\item Understand and implement various ameliorations (for example using
	a stateful backtracking algorithm \cite{spin08}, the
	``stack traversal" refinement \cite{popl05} or ``sleep sets" \cite{god95}),
	and evaluate performance improvement.
	
	\item Design and build a tool to visualise the execution of the
	backtracking state space exploration,
	perhaps even making a visual
	comparison between dynamic and static partial-order reduction on
	the same problem.
\end{itemize}

\section*{Plan of Work}

\subsubsection*{Michaelmas Weeks 3--6}

Study and understand relevant literature (especially the paper introducing dynamic partial-
order reduction \cite{popl05}). This is a significant part of this relatively
theory-heavy project. Specify syntax and semantics of a simple language for writing
concurrent software programs.

\subsubsection*{Michaelmas Weeks 7--8}

Implement, test and debug the language.

\subsubsection*{Christmas Vacation}

Design, write, test and debug the main structure of the system, in particular
the backtracking algorithm. The procedure that computes which backtracking
points are necessary is to be approximated by a function which suggests that every possible interleaving should
be examined.


\subsubsection*{Lent Weeks 1--3}

Implement, test and debug algorithm that decides which backtracking points are
necessary. Write and submit the required progress report.

\subsubsection*{Lent Weeks 4--5}

Evaluate project to show that the main success criterion has been met.

\subsubsection*{Lent Weeks 6--8}

If behind, catch up with timetable; otherwise, attempt
extension tasks. Begin writing first draft of dissertation.

\subsubsection*{Easter Vacation}

Continue work on dissertation. Deliver full draft to supervisors and
director of studies. Proofread and submit dissertation before beginning
of Easter Term.

\section*{Resources Required}

For this project, I shall mainly use my own computer,
an Acer laptop which runs Windows 10, has a 1.8 GHz dual-core processor,
and 6 GB of memory. In the event of its failure, I will use
the university's MCS machines. To mitigate the risk of loss of data,
I will work in a Dropbox folder, so that my files are automatically
backed up whenever my computer is connected to the Internet.
I will also make at least weekly backups to both OneDrive
and my university filespace.

\begin{thebibliography}{9}
	
	\bibitem{cgmp98}
	\href{https://www.cs.cmu.edu/~emc/15-820A/reading/partial-order.pdf}{
		E. M. Clarke, O. Grumberg, M. Minea and D. Peled,
		\emph{State Space Reduction using Partial Order Techniques},
		International Journal on Software Tools for Technology Transfer,
		1998.
	}
	
	\bibitem{popl05}
	\href{https://users.soe.ucsc.edu/~cormac/papers/popl05.pdf}{
		C. Flanagan and P. Godefroid,
		\emph{Dynamic Partial Order Reduction
			for Model Checking Software},
		Proceedings of the 32nd Symposium on Principles of Programming Languages,
		2005.
	}
	
	\bibitem{spin08}
	\href{http://www.cs.utah.edu/~yuyang/papers/spin08.pdf}{
		Y. Yang, X. Chen, G. Gopalakrishnan and R. M. Kirby,
		\emph{Efficient Stateful Dynamic Partial Order Reduction},
		Proceedings of the 15th International Workshop on Model Checking Software,
		2008
	}
	
	\bibitem{god95}
	\href{http://link.springer.com/book/10.1007%2F3-540-60761-7}{
		P. Godefroid,
		\emph{Partial-Order Methods for the Verification of
			Concurrent Systems -- An Approach to the State-Explosion
			Problem},
		 Volume 1032 of Lecture Notes in Computer
		 Science, Springer-Verlag,
		 1996
	}

\end{thebibliography}

\end{document}
