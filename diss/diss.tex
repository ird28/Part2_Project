% Isaac Dunn Part II Computer Science Dissertation
\documentclass[12pt,a4paper,twoside,openright]{report}
\usepackage[pdfborder={0 0 0}]{hyperref}    % turns references into hyperlinks
\usepackage[margin=25mm]{geometry}  % adjusts page layout
\usepackage{graphicx}  % allows inclusion of PDF, PNG and JPG images
\usepackage{verbatim}
\usepackage{docmute}   % only needed to allow inclusion of proposal.tex

\raggedbottom                           % try to avoid widows and orphans
\sloppy
\clubpenalty1000%
\widowpenalty1000%

\renewcommand{\baselinestretch}{1.1}    % adjust line spacing to make
                                        % more readable

\begin{document}

\bibliographystyle{plain}


%%%%%%%%%%%%%%%%%%%%%%%%%%%%%%%%%%%%%%%%%%%%%%%%%%%%%%%%%%%%%%%%%%%%%%%%
% Title page


\pagestyle{empty}

\rightline{\LARGE \textbf{Isaac Dunn}}

\vspace*{60mm}
\begin{center}
\Huge
\textbf{Dynamic Partial-Order Reduction for Model Checking} \\[5mm]
Computer Science Tripos -- Part II \\[5mm]
Clare College \\[5mm]
\today  % today's date
\end{center}

%%%%%%%%%%%%%%%%%%%%%%%%%%%%%%%%%%%%%%%%%%%%%%%%%%%%%%%%%%%%%%%%%%%%%%%%%%%%%%
% Proforma, table of contents and list of figures

\pagestyle{plain}

\chapter*{Proforma}

{\large
\begin{tabular}{ll}
Name:               & \bf Isaac Dunn                            			 \\
College:            & \bf Clare College                    				     \\
Project Title:      & \bf Dynamic Partial-Order Reduction for Model Checking \\
Examination:        & \bf Computer Science Tripos -- Part II, June 2016      \\
Word Count:         & \bf N    \\
Project Supervisors:& Dr. Jonathan Hayman \& Prof. Glynn Winskel             \\ 
\end{tabular}
}


\section*{Original Aims of the Project}

To do.

\section*{Work Completed}

To do.

\section*{Special Difficulties}

To do.
 
\newpage
\section*{Declaration}

I, Isaac Dunn of Clare College, being a candidate for Part II of the Computer
Science Tripos, hereby declare
that this dissertation and the work described in it are my own work,
unaided except as may be specified below, and that the dissertation
does not contain material that has already been used to any substantial
extent for a comparable purpose.

\bigskip
\leftline{Signed:}

\bigskip
\leftline{Date:}

\tableofcontents

\listoffigures

\newpage
\section*{Acknowledgements}

Put acknowledgements here.

%%%%%%%%%%%%%%%%%%%%%%%%%%%%%%%%%%%%%%%%%%%%%%%%%%%%%%%%%%%%%%%%%%%%%%%
% now for the body

\pagestyle{headings}

\chapter{Introduction}

\section{Model Checking}
We want to be very sure that some computer
programs are correct. Testing is not good
enough -- all software products have been
tested, but according to Dijkstra,
\emph{``Testing shows the presence,
	not the absence, of bugs''}.

Model checking is a category of formal
methods, in which...

\section{The State Explosion Problem}
Concurrent software programs in practice often
use threads as their model. The number of possible
interleavings of concurrent threads blows up very 
quickly. Give some maths and a picture showing this.

\section{Partial-Order Reduction}
Idea: some interleavings are equivalent. If you
can identifying which are equivalent, you only
have to explore a subset. Give picture.
Partial-order reduction is a category of techniques
that aim to do this.

\section{Dynamic Partial-Order Reduction}
DPOR uses runtime information to identify
equivalent interleavings. In fact, it assumes
that all interleavings are equivalent, picks
one arbitrarily, and takes note of backtracking
points which might lead to un-equivalent
interleavings.

\section{This Project}
The aim of this project is to implement DPOR.
It succeeded, and went further.

\chapter{Preparation}

\section{Background Definitions}


\section{Simple Model Checking}

\section{Dynamic Partial-Order Reduction}


\chapter{Implementation}

\section{Development of Project Language (PL)}
\subsection{Syntax and Semantics}
First design a simple but fairly practical sequential language.
$(e, s) \longrightarrow^t (e', s')$
Then using these as threads, add a global store, and read and CAS
operations on that store.
$(e, s, g) \longrightarrow^t (e', s', g)$
A program consists of multiple $(e, s)$ threads, and one global store,
$g$, and at any given time any of the threads can advance.

See appendix for language definition.

\subsection{Parser}
Use <tool> and <tool> to specify the conversion of source code
into abstract expressions. Example?

\subsection{Interpreter}
Auxiliary functions for things like substitution, deciding if
an expression is a value, modifying de Bruijn indices.

Stores simply implemented as (location, value) pairs.

Function for getting next local step of thread, given global store.

Function for getting the next transition of a thread; all the local
steps up to and including a visible operation.

\section{Simple Model Checker}

Depth first search of state space. Present algorithm used.
Explain how errors and deadlocks are found.

\section{Dynamic Partial-Order Reduction}

\subsection{Clock Vectors}
\subsection{Locks}
Problem: many algorithms use locks. While \texttt{while locked do skip done}
is good enough, it results in a cycle in the state space, which causes
non-termination of DPOR.

Solution: add locks. New problem: no distinction between
transition not existing and not being enabled. Without
locks, all existing transitions were enabled; no longer
the case.

Solution: pass around Boolean stating whether or not the transition
is enabled.

\section{Stateful Model Checking}

\subsection{Simple Model Checker}
When searching through state space, the same state is often
re-encountered. Idea: keep track of visited states, and don't bother
re-exploring from those we have already explored.

\subsection{Dynamic Partial-Order Reduction}

\subsubsection{Na\"{\i}ve Approach}
The na\"{\i}ve approach is unsound.
Give explanation.

\subsubsection{Stateful Dynamic Partial-Order Reduction}
Idea: conservatively add in necessary points to backtrack sets.
Use visible operation dependency graph.

\chapter{Evaluation}

\section{Soundness of Model Checking}

\begin{itemize}
	\item Evidence of correctness of language implementation
	\item Example programs \& comparison to simple checker
\end{itemize}

\section{Efficiency of Model Checking}

\begin{itemize}
	\item Example programs
	\item Performance tables/graphs
\end{itemize}

\chapter{Conclusions}

This chapter is empty!


%%%%%%%%%%%%%%%%%%%%%%%%%%%%%%%%%%%%%%%%%%%%%%%%%%%%%%%%%%%%%%%%%%%%%
% the bibliography
\addcontentsline{toc}{chapter}{Bibliography}
\bibliography{refs}

%%%%%%%%%%%%%%%%%%%%%%%%%%%%%%%%%%%%%%%%%%%%%%%%%%%%%%%%%%%%%%%%%%%%%
% the appendices
\appendix

\chapter{First appendix}
 
 This appendix is empty!
 
\chapter{Second appendix}

This appendix is empty!

\chapter{Project Proposal}

\input{proposal_without_title}

\end{document}
